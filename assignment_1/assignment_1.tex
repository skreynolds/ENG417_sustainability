\documentclass[11pt]{article}
\usepackage[utf8]{inputenc}	% Para caracteres en español
\usepackage{amsmath,amsthm,amsfonts,amssymb,amscd}
\usepackage{multirow,booktabs}
\usepackage[table]{xcolor}
\usepackage{fullpage}
\usepackage{lastpage}
\usepackage{enumitem}
\usepackage{fancyhdr}
\usepackage{mathrsfs}
\usepackage{wrapfig}
\usepackage{setspace}
\usepackage{calc}
\usepackage{multicol}
\usepackage{cancel}
\usepackage[margin=3cm]{geometry}
\usepackage{amsmath}
\newlength{\tabcont}
\setlength{\parindent}{0.0in}
\setlength{\parskip}{0.05in}
\usepackage{framed}
\usepackage{tcolorbox}
\usepackage{xcolor}
\usepackage{graphicx}
%\usepackage{epstopdf}
%\epstopdfDeclareGraphicsRule{.gif}{png}{.png}{convert gif:#1 png:\OutputFile}
\colorlet{shadecolor}{orange!15}
\parindent 0in
\parskip 12pt
\geometry{margin=1in, headsep=0.25in}
\theoremstyle{definition}
\newtheorem{defn}{Definition}
\newtheorem{reg}{Rule}
\newtheorem{exer}{Exercise}
\newtheorem{note}{Note}

\graphicspath{{./fig/}}

\begin{document}
\setcounter{section}{0}
\title{Chapter 9 Review Notes}

\thispagestyle{empty}

\begin{center}
{\LARGE \bf Role-play 1 Notes}\\
{\large ENG417 Sustainability}\\
\end{center}
\section*{Has mining investment slowed in Australia? Has there been a recovery in commodity prices?}
The Chinese economy experienced increasing GDP growth rates over an 8 year period from 2000 through to late 2007, as shown in Figure 1. During this time, increased economic activity in China resulted in higher demand for natural resources, specifically iron ore. The mining sector of the Australia economy benefited during this period with higher levels of investment, as shown in Figure 2.\\
\begin{figure}[h]
	\begin{minipage}[t]{0.57\textwidth}
		\centering
		\boxed{\includegraphics[height=4.6cm]{china-gdp-growth-annual}}
		\caption{GDP growth in China}
	\end{minipage}
	\begin{minipage}[t]{0.56\textwidth}
		\centering
		\boxed{\includegraphics[height=4.6cm]{20170311_ASC117}}
		\caption{Mining investment in Australia}
	\end{minipage}
\end{figure}

It was a global economic downturn, which was the result of the Global Financial Crisis (GFC), which reversed the gains in China's GDP growth rates. The GFC did not immediately impact the Australian mining sector. This is evident in two economic measures:
\vspace{-0.5cm}
\begin{enumerate}
	\item The stability of commodity prices during the GFC, and for a number of years afterwards, seen in Figure 3
	\item The continued investment in mining projects after the GFC, as shown in Figure 2
\end{enumerate}

\begin{figure}[h]
	\centering
	\boxed{\includegraphics[scale=0.3]{iron_ore}}
	\caption{Iron ore prices in USD}
\end{figure}

It is important to note that these figures show that the mining sector continued to show economic strength until late 2012. At the time of writing these notes, it is apparent that mining investment has fallen dramatically. The same can be said for commodity prices.

\section*{What impact have these changes had on Australian GDP?}
GDP growth, while it is an imperfect measure, can be used to estimate the economic activity of a country. Typically, a country will experience some GDP inflation due to factors like population growth. It is widely accepted, however, that GDP growth beyond inflation is a desirable characteristic representative of increasing living standards. The aggregate GDP growth for the global economy can be seen in Figure 4 - at present global GDP growth remains static at around 4\%.

\begin{figure}[h]
	\centering
	\includegraphics[height=6cm]{gdp-growth-world}
	\caption{Global GDP growth}
\end{figure}

Against this backdrop of global GDP growth it's easy to think that Australia has been under performing in recent times - the current Australian GDP growth of 2\% can be seen in Figure 5. However, this is a spurious claim since emerging economies often experience much higher rates of growth which would distort the aggregate growth figure. A fairer comparison would be with other advanced economies. Figure 6 shows that, on average, advanced economies around the world are experiencing GDP growth rates of about 2\%. Viewed in the context of other advanced economies, Australia's GDP growth rate does not seem alarming.

\begin{figure}[h]
	\begin{minipage}[t]{0.5\textwidth}
		\centering
		\includegraphics[height=5cm]{gdp-growth}
		\caption{GDP growth in Australia}
	\end{minipage}
	\begin{minipage}[t]{0.5\textwidth}
		\centering
		\includegraphics[height=5cm]{gdp-growth-advanced-economies}
		\caption{GDP growth in OECD nations}
	\end{minipage}
\end{figure}
 
So why hasn't the reduction in mining investment and the fall in commodity prices had a more pronounced effect on the Australian GDP growth rate? Any introductory textbook in Macroeconomics provides us with some insight - very roughly, GDP growth can be defined by the relationship
\begin{align*}
	Y = C + I + G + X,	
\end{align*}
where $Y$ is GDP, $C$ is consumption, $I$ is investment, $G$ is government spending, and $X$ is net exports. Ameliorating some of the reduction in the GDP growth rate due to a fall in mining investment was an increase in mining exports. Very simply, the sector underwent a rebalancing in the type of economic activity that was occurring, as seen in Figure 7.\\

It could be argued that the Australian economy has not been adversely affected by the downturn in mining investment, thanks to an increase in exports. Furthermore, the marginal fall in GDP growth put Australia in line with other advanced economies.

Perhaps the cause for concern is ill-founded?

It must be noted that these economic measures are all backward looking and as such can only tell us of our historical performance. In fact, very little can be analytically derived about the future of the Australian economy from these measures. It is the role of speculation to help us form ideas of what may happen in the future. For example, extrapolating a line of best fit to Figure 5 may create the belief that Australian GDP growth will continue a concerning downward trend, ceteris paribus.
\begin{figure}[h]
	\centering
	\boxed{\includegraphics[scale=0.25]{ore_exports}}
	\caption{Iron ore exports in value and volume}
\end{figure}
\newpage
\section*{How will this impact Australian schools, specifically science, technology, engineering, and mathematics?}
Although the economy as a whole has not been widely affected, it's not difficult to imagine reduced mining investment dampening the need for science, technology, engineering, and mathematics (STEM) skilled professionals. Indeed, the fall in engineering job vacancies, shown in Figure 8, seems to correlate strongly with the fall in mining investment seen in Figure 2.
\begin{figure}[h]
	\centering
	\boxed{\includegraphics[scale=0.5]{engineering_job_vac}}
	\caption{Engineering vacancy rate}
\end{figure}

But is this part of a broader trend? Figure 9 shows a marked decline in engineering investment across all business sectors, indicating that there may be less demand, in a more general sense, for STEM skilled professionals in engineering roles.
\begin{figure}[h]
	\centering
	\includegraphics[scale=0.5]{business-investment-components}
	\caption{Business investment in Australia}
\end{figure}

At the same time the supply of domestic engineering graduates has continued to increase at a somewhat constant rate of growth, seen in Figures 10 and 11. Further to this, until recently, skilled migration of engineers had been contributing significantly to the pool of talent, as shown in Figure 12. This is an environment in which there is a falling general demand for STEM skilled professionals in engineering roles, and a static supply of engineering graduates - a perfect recipe for oversupply and increased unemployment. This can be seen in Figure 13. 

\begin{figure}[h]
	\begin{minipage}{0.5\textwidth}
		\centering
		\includegraphics[scale=0.5]{entry}
		\caption{Undergraduate engineering degree completions}
	\end{minipage}
	\begin{minipage}{0.5\textwidth}
		\centering
		\includegraphics[scale=0.5]{postgrad}
		\caption{Postgraduate engineering degree completions}
	\end{minipage}
\end{figure}

\begin{figure}[h]
	\begin{minipage}{0.5\textwidth}
		\centering
		\boxed{\includegraphics[height=4.5cm]{migration}}
		\caption{Skilled migration for engineers to Australia}
	\end{minipage}
	\begin{minipage}{0.5\textwidth}
		\centering
		\boxed{\includegraphics[height=4.5cm]{eng_profession}}
		\caption{Engineering professions job market}
	\end{minipage}
\end{figure}

Fortunately, widespread unemployment of STEM professionals is not a reality. Figure 14 shows that supply and demand for engineering graduates are closely matched, albeit a large majority of engineers are not working in engineering fields. Moreover, there has been a recent downturn in traditional engineering occupations. The important takeaway from this figure is that industry places a high value on professionals with STEM qualifications.

\begin{figure}[h]
	\centering
	\includegraphics[scale=0.4]{supply_demand}
	\caption{Supply and demand for engineering skill sets}
\end{figure}

In light of the value that industry places on professionals with STEM skills, it stands to reason that student engagement in STEM subjects, as a percentage of total enrolment, would be growing or at the very least, remaining constant. This, however, is not the case. Figures 15, 16, 17, and 18 show that engagement in STEM subjects has been on the decline for a decade. Interestingly, this has not yet affected the rate at which universities are producing engineering graduates, seen in Figures 10 and 11.

\begin{figure}[h]
	\begin{minipage}{0.5\textwidth}
		\centering
		\includegraphics[height=4.5cm]{int_math}
		\caption{School performance in intermediate mathematics}
	\end{minipage}
\begin{minipage}{0.5\textwidth}
	\centering
	\includegraphics[height=4.5cm]{adv_math}
	\caption{School perfomance in advanced mathematics}
\end{minipage}
\end{figure}

\begin{figure}[h]
	\begin{minipage}{0.5\textwidth}
		\centering
		\includegraphics[height=4.5cm]{chem}
		\caption{School perfomance in chemistry}
	\end{minipage}
	\begin{minipage}{0.5\textwidth}
		\centering
		\includegraphics[height=4.5cm]{physics}
		\caption{School perfomance in physics}
	\end{minipage}
\end{figure}

To summarise, the decrease in mining investment has played a minimal part in the decrease in student participation in STEM based subjects. Specifically, the down turn mining investment is consistent with a much broader move in the Australian economy away from engineering investment. This, however, has not led to widespread unemployment of engineers who are readily able to get positions in other fields. Reason suggests that this should provide extrinsic motivation for students to develop STEM skills, however, there has been a decline in student participation in STEM related subjects for over a decade. Paradoxically, this has not (yet) led to a decrease in the rate of engineering graduates.
\newpage

\section*{Proposal for the re-invigoration of the Australian economy?}
Previously it was established that the Australian economy is not currently suffering from the decrease in mining investment, or the fall in commodity prices. However, the GDP growth rate has been trending downwards for a number of years. If, hypothetically speaking, Australia's declining GDP growth rate were to continue, there are a number of ways which Australia could address this.

\subsection*{What role could innovation, research, and development play?}
In the late 90s, the Israel government significantly increased spending (as a percentage of GDP) on research and development, as shown in Figure 19. This two decade long, sustained level of investment in this field has begun to spawn an increasing number of firms operating in the high-tech sector - the Economist reports that there are over 1000 start ups per year.

\begin{figure}[h]
	\centering
	\includegraphics[scale=0.3]{israel_rd}
	\caption{text}
\end{figure}

The Israeli ministry of foreign affairs reports that output from the high-tech sector now accounts for around 70\% of the total economic output of their industrial sector, of which high-tech is a part. Further, a reported 80\% of high-tech products are exported to foreign markets. GDP growth in Israel can be seen in in Figure 20. The figure shows that the GDP growth is loosely correlated with the increase in R\&D spending, and whilst there is an obvious increase in the the volatility of GDP growth, it has remained high (by OECD standards) throughout this duration.

\begin{figure}[h]
	\centering
	\boxed{\includegraphics[scale=0.2]{israel_gdp}}
	\caption{text}
\end{figure}

This is one avenue through with Australia may be able to contribute to its GDP growth rate. Culturally, Australia may be very different to Israel, however, in many respects Australia may be better positioned for research and innovation growth. Figure 21 shows a side by side comparison of the performance of the national science and innovation system of Australia and Israel. Australia outranks Israel in terms of young talent in science education, doctoral graduates from science and engineering, ease of entrepreneurship, and internet infrastructures for innovation. Furthermore, it could be argued that Israel outranks Australia in aspects of venture capital flow, and private sector R\&D spending as a direct result of increased government spending. 

\begin{figure}[h]
	\centering
	\boxed{\includegraphics[scale=0.5]{australia_compare}}
	\caption{text}
\end{figure}

Australia has the added benefit that many of its existing industries are mature and stable, unlike Israel whose other industries are largely unprotected from competition and inefficient, according to the Economist. This would help to avoid excessive volatility seen in the Israel GDP growth rate.

%\subsection*{What role could advanced manufacturing play?}

%\subsection*{What role could the processing and storage of nuclear waste play?}
\newpage
\subsection*{How would increased spending on research and development impact Australian schools, specifically, science, technology, engineering, and mathematics?}

Increased spending on research and development could be seen to provide a greater number of employment opportunities for professionals with STEM skills. This in turn could provide students with higher levels of extrinsic motivation helping to boost engagement in STEM related subjects. This would mean an increase in the demand for educational professionals with a specialisation in STEM. Moreover, there would also be greater opportunity for improvements in pedagogy, higher teaching quality, and more scope for innovative teaching practices.

In addition to the impact on STEM in schools, the following benefit may also flow from the increased R\&D investment:
\begin{enumerate}
	\item An increased level of private spending on research and development further supporting the high-tech industry;
	\item A return of manufacturing to Australia, specifically, increased activity in developing advanced manufacturing processes, and the production of goods with high margins (similar to Germany);
	\item Increased activity in the some services, like Accounting, Legal, Finance, and Marketing - the high-tech industry does not operate in a vacuum. Business relies on support services;
	\item A tighter coupling with University research and the needs of corporations, leading to higher levels of commercialisation of technologies;
	\item An increased amount of publications from Australian Universities in top tier academic journals, leading to higher ranked Australian Universities. This in turn could result in the export of more educational services as the perceived quality of education increases. This would further strengthen an already important industry;
\end{enumerate}

\section*{Summary}
It must be acknowledged that the Australian mining sector has experienced a decrease in investment, and that commodity prices have weakened significantly, however, the Australian economy has not experienced a significant downturn due to increased export activity from the sector. Speculatively, it could be asserted that the Australian economy may experience increasingly lower GDP growth rates in future years.

This decrease in mining investment has not impacted the number of STEM professionals employed. This decrease in engineering investment is part of a broader trend in Australian business investment. Recently, there has been a downturn in employment opportunities for STEM professionals in traditional engineering roles, however, this has not led to widespread unemployment for these professionals. Engineering degrees are highly valued allowing access to many non-traditional engineering roles.

The high levels of employment enjoyed by STEM professionals does not translate to increased levels of participation in STEM subjects in high schools. In fact, there is a decade long decline in participation rates for mathematics and science seen in Australian Schools. This may be in part a perceived tenuousness in the link between engineering and non-traditional engineering roles.

Should the Australian economy continue it's lower GDP growth trend, one course of action which may help to address this could be found in increased government spending on research and development. This has been successfully implemented by small countries like Israel and despite some cultural dissimilarities, Australia may be better placed to take advantage of these opportunities.

\end{document}